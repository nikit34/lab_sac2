\documentclass[14pt, a4paper]{article}
\usepackage[left=2cm, right=1cm, top=2cm, bottom=2cm]{geometry}
\usepackage[utf8x]{inputenc}
\usepackage[T2A]{fontenc}
\usepackage[english,russian]{babel}
\usepackage{hyperref}
\usepackage{sagetex}
\usepackage{blindtext}
\setlength{\sagetexindent}{10ex}

\title{\quad\quad\quad\quadЛабораторная работа №5 по курсу
\newline "Системы аналитических вычислений"}

\author{Выполнил: Пермяков Никита
\and М8О-208Б-19
\and задание 3}

\begin{document}
\maketitle

\section{Приведение уравнения поверхности второго рода к каноническому виду}


\begin{sagesilent}
    # Точность вычислений (знаков после запятой)
    accurcity = 10

    # Границы для трёхмерного графика
    plot_range = 10.0

    u(x, y, z) = 6*x^2 + 12*x*y + 7*y^2 + 2*x*z + 3*z^2 + 5*x + 5*y + 5*z - 18
\end{sagesilent}

Исходное уравнение поверхности второго рода:

$\sage{u(x, y, z)} = 0$

\begin{sagesilent}
    # Из уравнения поверхности получим коэффициенты
    a11 = 6
    a22 = 7
    a33 = 3

    a12 = 6
    a13 = 1
    a23 = 0

    a0 = -18
    a1 = 5 / 2
    a2 = 5 / 2
    a3 = 5 / 2

    A = matrix([
        [a11, a12, a13],
        [a12, a22, a23],
        [a13, a23, a33]
    ])

    a = vector([a1, a2, a3])
\end{sagesilent}

Матричный вид уравнения:

$\sage{LatexExpr("(A|a) = " + latex(A) + latex(a) + "^T")}$

И вектор линейных коэффициентов:

$\sage{LatexExpr("a = " + latex(a))}$

\begin{sagesilent}
    # Характеристический многочлен
    var("z")
    detA = A - z * identity_matrix(3) 
    detA = det(detA).simplify_full()

\end{sagesilent}

Составим характеристический многочлен:

$\sage{LatexExpr(r'\det(A - z * E) = ' + latex(detA))}$

\begin{sagesilent}
    # Собственные числа
    roots = solve(detA == 0, z)
    eigens = []

    for i, root in enumerate(roots):
        if (root.rhs().imag().n(digits = accurcity) < 10 ** -accurcity):
            eigens.append(root.rhs().real().n(digits = accurcity))
\end{sagesilent}

Корни характеристического уравнения:

$\sage{LatexExpr("z_0 = " + str(eigens[0]) + r', z_1 = ' + str(eigens[1]) + r', z_2 = ' + str(eigens[2]))}$



Проверка через метод
$\sage{LatexExpr(latex(A.eigenvalues()))}$


\begin{sagesilent}
    # Собственные вектора
    eigenvectors = []
    var("x1 x2 x3")

    for i, eigen in enumerate(eigens):
        dotA = A - eigen * identity_matrix(3)    
        equations = []
        for j in range(3):
            equations.append(dotA[j][0] * x1 + dotA[j][1] * x2 + dotA[j][2] * x3 == 0)

        equations[1] = equations[1] - equations[0] * dotA[1][0] / dotA[0][0]
        equations[2] = equations[2] - equations[0] * dotA[2][0] / dotA[0][0]
        equations[2] = equations[2] - equations[1] * (equations[2].lhs() / equations[1].lhs())

        if (equations[2].lhs() == 0 and equations[2].rhs() == 0):
            equations[2] = (x3 == 1)
        else:
            print("Ранг матрицы равен трём")
        eigenvec = vector([j.rhs().n(digits = accurcity) for j in solve(equations, x1, x2, x3)[0]])
        ans = dotA * eigenvec
        eigenvectors.append(eigenvec.n(digits = accurcity))
\end{sagesilent}

Получим собственные векторы:

$\sage{LatexExpr(r'\lambda_0 = ' + str(eigens[0].n(digits = accurcity)) + ', s_0 = ' + str(eigens[0].n(digits = accurcity)))}$

$\sage{LatexExpr(r'\lambda_1 = ' + str(eigens[1].n(digits = accurcity)) + ', s_1 = ' + str(eigenvec[1].n(digits = accurcity)))}$

$\sage{LatexExpr(r'\lambda_2 = ' + str(eigens[2].n(digits = accurcity)) + ', s_2 = ' + str(eigenvec[2].n(digits = accurcity)))}$

\begin{sagesilent}
    # Получим матрицу перехода
    S = matrix(eigenvectors)
\end{sagesilent}


Проверка через метод
$\sage{LatexExpr(latex(A.eigenvectors_right()))}$

Получим матрицу перехода к диагональной матрице

$\sage{LatexExpr("S = " + latex(S.n(digits = accurcity)))}$

\begin{sagesilent}
    # Нормируем собсветнные векторы
for i in range(len(eigenvectors)):
    eigenvectors[i] = eigenvectors[i] / sqrt(eigenvectors[i].dot_product(eigenvectors[i]))
    normirS = matrix(eigenvectors)
\end{sagesilent}

После нормировки собственных векторов матрица перехода имеет вид:

$\sage{LatexExpr("S^* = " + latex(normirS.n(digits = accurcity)))}$

\begin{sagesilent}
    diagA = normirS * A * normirS.T
\end{sagesilent}

Приводя к диагональному виду:

$\sage{LatexExpr("S^{*} * A * {S^{*}}^T = " + latex(diagA.n(digits = accurcity)))}$

\begin{sagesilent}
    a1 = normirS.T * a
\end{sagesilent}

Новый вектор линейных коэффициентов:

$\sage{a1}$

\begin{sagesilent}
    v(x, y, z) = 0
    variables = [x, y, z]
    for i in range(len(variables)):
        v += diagA[i][i] * variables[i] ^ 2 + variables[i] * a1[i]
\end{sagesilent}

После приведения к каноническому виду получим:

$\sage{v(x, y, z) == 0}$

\begin{sagesilent}
    # Сделаем замену переменных
    coef_linear = []
    for i in range(len(a1)):
        coef_linear.append((a1[i] / (2 * diagA[i][i])) ** 2)
        if (diagA[i][i] * a1[i] < 0):
            coef_linear[-1] = -coef_linear[-1]
    v(x, y, z) = a0
    variables = [x, y, z]
    for i in range(len(variables)):
        v += diagA[i][i] * (variables[i] + coef_linear[i]) ^ 2 - diagA[i][i] * abs(coef_linear[i])
\end{sagesilent}

Сделаем замену переменных, получим:

$\sage{v(x, y, z) == 0}$

\pagebreak

Сравним графики исходного уравнения поверхности второго рода и канонического уравнения после замены переменных.

График функции $\sage{u(x, y, z)} = 0$
\begin{center}
  \sageplot[trim=50 50 50 50, clip, width=10cm][png]{implicit_plot3d(u, (-plot_range, plot_range), (-plot_range, plot_range), (-plot_range, plot_range), color="orange")}
\end{center}

График функции $\sage{v(x, y, z)} = 0$
\begin{center}
  \sageplot[trim=50 50 50 50, clip, width=10cm][png]{implicit_plot3d(v, (-plot_range, plot_range), (-plot_range-26, plot_range-26), (-plot_range, plot_range), color="orange")}
\end{center}

\end{document}